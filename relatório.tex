% !TeX spellcheck = pt_PT
\documentclass[12pt,a4paper]{article}
\usepackage[portuguese]{babel}
\usepackage[utf8]{inputenc} 
\usepackage{natbib}
\author{André Cirne e José Sousa}
\title{Aprendizagem máquina e o ID3}
\begin{document}
	
\begin{titlepage}
	\centering
	{\scshape\LARGE Faculdade de Ciências \par}
	\vspace{1cm}
	{\scshape\Large Inteligência Artificial\par}
	\vspace{1.5cm}
	{\huge\bfseries Trabalho 4\par}
	\vspace{2cm}
	{\Large\itshape André Cirne e José Sousa\par}
	\vfill
	
	{\large \today\par}
\end{titlepage}

\tableofcontents
\section{Introdução}
Uma árvore de decisão é a representação de uma função que efetua a avaliação a um determinado conjunto de dados e que consegue retornar uma decisão \cite{stuart2016artificial}, com base em conhecimento que foi previamente dado à máquina. Este tipo de representação de dados é aquele que permite uma mais simples e maior taxa de sucesso quando utilizados em conjunção com algoritmos de indução em árvores de decisão.
Numa árvore de decisão cada nó representa um teste ao conjunto de dados que se pretende avaliar e cada um dos seus ramos encontra-se identificado com os possíveis valores que pode tomar. Estes testes e os seus ramos vão conduzindo o algoritmo até a um nó folha aonde se encontra o valor a ser retornado pela função, que será a sua decisão.

\section{Algoritmos de indução de árvores de decisão}
As árvores de decisão são apenas uma estrutura de dados, logo sem algoritmos capazes de construir estes modelos de previsão de forma eficiente, elas não servem para nada.
Os algoritmos de indução de árvores de decisão são algoritmos que dados o conjunto de dados de exemplo consegue representar de forma genérica as relações existentes entre dados de input e os resultados.Que caso especifico usando essas relações constroem de forma automática a árvore de decisão.\cite{rokach2005top} Estes algoritmos por norma têm uma construção aplicando princípios \textit{greedy} e construido a árvore na direção raiz para as folhas de forma recursiva.
Como o referido anteriormente este algoritmos são de sua essência greedy logo para que em cada nó ele possa efetuar a melhor escolha vai utilizar métricas.
\subsection{Métricas}
\subsubsection{Ganho}

\subsubsection{Impuridade de Gini}
A impuridade Gini

\subsection{ID3}

\subsection{CART}

\subsection{C4.5}

\section{Implementsção}
Neste trabalho utilizamos Python 3, devido a ser uma linguagem multi-paradigma e levando assim a uma passagem mais rápida da parte do planeamento para implentação.
\subsection{Estrutura de dados}

\subsubsection{Dicionários}

\subsubsection{Listas}

\subsubsection{Leaf}

\subsubsection{Jump}

\subsubsection{Node_root}

\subsection{Organização do código}

\subsection{Reflexão}
Durante a fase de implemen

\bibliographystyle{plain}
\bibliography{bibliografia} 

\end{document}


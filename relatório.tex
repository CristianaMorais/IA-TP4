% !TeX spellcheck = pt_PT
\documentclass[12pt,a4paper]{article}
\usepackage[portuguese]{babel}
\usepackage[utf8]{inputenc} 
\usepackage{natbib}
\author{André Cirne e José Sousa}
\title{Aprendizagem máquina e o ID3}
\begin{document}
	
\begin{titlepage}
	\centering
	{\scshape\LARGE Faculdade de Ciências \par}
	\vspace{1cm}
	{\scshape\Large Inteligência Artificial\par}
	\vspace{1.5cm}
	{\huge\bfseries Trabalho 4\par}
	\vspace{2cm}
	{\Large\itshape André Cirne e José Sousa\par}
	\vfill
	
	{\large \today\par}
\end{titlepage}

\tableofcontents
\section{Introdução}
Uma árvore de decisão é a representação de uma função que efetua a avaliação a um determinado conjunto de dados e que consegue retornar uma decisão \cite{stuart2016artificial}, com base em conhecimento que foi previamente dado à máquina. Este tipo de representação de dados é aquele que permite uma mais simples e maior taxa de sucesso quando utilizados em conjunção com algoritmos de indução em árvores de decisão.
Numa árvore de decisão cada nó representa um teste ao conjunto de dados que se pretende avaliar e cada um dos seus ramos encontra-se identificado com os possíveis valores que pode tomar. Estes testes e os seus ramos vão conduzindo o algoritmo até a um nó folha aonde se encontra o valor a ser retornado pela função, que será a sua decisão.

\section{Algoritmos de indução de árvores de decisão} \cite{@nyan#42722017discord}

\bibliographystyle{plain}
\bibliography{bibliografia} 

\end{document}

